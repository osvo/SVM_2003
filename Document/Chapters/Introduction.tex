\chapter{Introduction}\label{ch:introduction}

Now, perhaps more than ever, it is necessary to build trust in the structures since there is skepticism founded on several recent structural collapses although they are still rare events. Structural reliability is thus an area of study that has the potential to save lives. However, to quantify the structural reliability, the probability of failure \(P_f\) must be calculated and is usually done through the \ac{MCM} which has the disadvantage that it has a high computational demand.

Not in vain have more efficient alternatives been sought since the early years of the 21st century as polynomial response surfaces adjusted to a design of experiments, chaotic polynomials, multilayer perceptrons, among others. \defcitealias{Hurtado2003}{Hurtado and Álvarez (2003)} \citetalias{Hurtado2003}\citep{Hurtado2003} proposed a method that employs \ac{SVM}, in order to convert the computation of the probability of failure into a classification task. In this work, the performance of that algorithm will be verified and compared with the aforementioned Monte Carlo.

\section{Motivation}\label{sec:motivation}

Use the concepts learned in the areas of structural reliability, \ac{MCS} and Support Vector Machines (\ac{SVM}) in an eventual master's degree. Additionally and keeping the proportions, have a first experience in writing a document of this type together with its respective research and implementation.

\section{Objective}

The idea in this paper is to review some necessary concepts in structural reliability, Monte Carlo simulations and support vector machines.

The python programming language (version 3.8.8) with some libraries such as numpy, matplotlib and sklearn will be used for Monte Carlo and \ac{SVM} algorithms of which their results will be compared.

\hfill

\vfill


\paragraph{Important Note:} The Python codes mentioned in this document as well as the source code of this document in \LaTeX\ can be consulted at the following link:
\begin{center}
	\faGithub \hspace{0.5cm}\url{https://github.com/osvo/SVM_2003}
\end{center}

